% Options for packages loaded elsewhere
\PassOptionsToPackage{unicode}{hyperref}
\PassOptionsToPackage{hyphens}{url}
\PassOptionsToPackage{dvipsnames,svgnames,x11names}{xcolor}
%
\documentclass[
  letterpaper,
  DIV=11,
  numbers=noendperiod]{scrreprt}

\usepackage{amsmath,amssymb}
\usepackage{iftex}
\ifPDFTeX
  \usepackage[T1]{fontenc}
  \usepackage[utf8]{inputenc}
  \usepackage{textcomp} % provide euro and other symbols
\else % if luatex or xetex
  \usepackage{unicode-math}
  \defaultfontfeatures{Scale=MatchLowercase}
  \defaultfontfeatures[\rmfamily]{Ligatures=TeX,Scale=1}
\fi
\usepackage{lmodern}
\ifPDFTeX\else  
    % xetex/luatex font selection
\fi
% Use upquote if available, for straight quotes in verbatim environments
\IfFileExists{upquote.sty}{\usepackage{upquote}}{}
\IfFileExists{microtype.sty}{% use microtype if available
  \usepackage[]{microtype}
  \UseMicrotypeSet[protrusion]{basicmath} % disable protrusion for tt fonts
}{}
\makeatletter
\@ifundefined{KOMAClassName}{% if non-KOMA class
  \IfFileExists{parskip.sty}{%
    \usepackage{parskip}
  }{% else
    \setlength{\parindent}{0pt}
    \setlength{\parskip}{6pt plus 2pt minus 1pt}}
}{% if KOMA class
  \KOMAoptions{parskip=half}}
\makeatother
\usepackage{xcolor}
\setlength{\emergencystretch}{3em} % prevent overfull lines
\setcounter{secnumdepth}{5}
% Make \paragraph and \subparagraph free-standing
\ifx\paragraph\undefined\else
  \let\oldparagraph\paragraph
  \renewcommand{\paragraph}[1]{\oldparagraph{#1}\mbox{}}
\fi
\ifx\subparagraph\undefined\else
  \let\oldsubparagraph\subparagraph
  \renewcommand{\subparagraph}[1]{\oldsubparagraph{#1}\mbox{}}
\fi

\usepackage{color}
\usepackage{fancyvrb}
\newcommand{\VerbBar}{|}
\newcommand{\VERB}{\Verb[commandchars=\\\{\}]}
\DefineVerbatimEnvironment{Highlighting}{Verbatim}{commandchars=\\\{\}}
% Add ',fontsize=\small' for more characters per line
\usepackage{framed}
\definecolor{shadecolor}{RGB}{241,243,245}
\newenvironment{Shaded}{\begin{snugshade}}{\end{snugshade}}
\newcommand{\AlertTok}[1]{\textcolor[rgb]{0.68,0.00,0.00}{#1}}
\newcommand{\AnnotationTok}[1]{\textcolor[rgb]{0.37,0.37,0.37}{#1}}
\newcommand{\AttributeTok}[1]{\textcolor[rgb]{0.40,0.45,0.13}{#1}}
\newcommand{\BaseNTok}[1]{\textcolor[rgb]{0.68,0.00,0.00}{#1}}
\newcommand{\BuiltInTok}[1]{\textcolor[rgb]{0.00,0.23,0.31}{#1}}
\newcommand{\CharTok}[1]{\textcolor[rgb]{0.13,0.47,0.30}{#1}}
\newcommand{\CommentTok}[1]{\textcolor[rgb]{0.37,0.37,0.37}{#1}}
\newcommand{\CommentVarTok}[1]{\textcolor[rgb]{0.37,0.37,0.37}{\textit{#1}}}
\newcommand{\ConstantTok}[1]{\textcolor[rgb]{0.56,0.35,0.01}{#1}}
\newcommand{\ControlFlowTok}[1]{\textcolor[rgb]{0.00,0.23,0.31}{#1}}
\newcommand{\DataTypeTok}[1]{\textcolor[rgb]{0.68,0.00,0.00}{#1}}
\newcommand{\DecValTok}[1]{\textcolor[rgb]{0.68,0.00,0.00}{#1}}
\newcommand{\DocumentationTok}[1]{\textcolor[rgb]{0.37,0.37,0.37}{\textit{#1}}}
\newcommand{\ErrorTok}[1]{\textcolor[rgb]{0.68,0.00,0.00}{#1}}
\newcommand{\ExtensionTok}[1]{\textcolor[rgb]{0.00,0.23,0.31}{#1}}
\newcommand{\FloatTok}[1]{\textcolor[rgb]{0.68,0.00,0.00}{#1}}
\newcommand{\FunctionTok}[1]{\textcolor[rgb]{0.28,0.35,0.67}{#1}}
\newcommand{\ImportTok}[1]{\textcolor[rgb]{0.00,0.46,0.62}{#1}}
\newcommand{\InformationTok}[1]{\textcolor[rgb]{0.37,0.37,0.37}{#1}}
\newcommand{\KeywordTok}[1]{\textcolor[rgb]{0.00,0.23,0.31}{#1}}
\newcommand{\NormalTok}[1]{\textcolor[rgb]{0.00,0.23,0.31}{#1}}
\newcommand{\OperatorTok}[1]{\textcolor[rgb]{0.37,0.37,0.37}{#1}}
\newcommand{\OtherTok}[1]{\textcolor[rgb]{0.00,0.23,0.31}{#1}}
\newcommand{\PreprocessorTok}[1]{\textcolor[rgb]{0.68,0.00,0.00}{#1}}
\newcommand{\RegionMarkerTok}[1]{\textcolor[rgb]{0.00,0.23,0.31}{#1}}
\newcommand{\SpecialCharTok}[1]{\textcolor[rgb]{0.37,0.37,0.37}{#1}}
\newcommand{\SpecialStringTok}[1]{\textcolor[rgb]{0.13,0.47,0.30}{#1}}
\newcommand{\StringTok}[1]{\textcolor[rgb]{0.13,0.47,0.30}{#1}}
\newcommand{\VariableTok}[1]{\textcolor[rgb]{0.07,0.07,0.07}{#1}}
\newcommand{\VerbatimStringTok}[1]{\textcolor[rgb]{0.13,0.47,0.30}{#1}}
\newcommand{\WarningTok}[1]{\textcolor[rgb]{0.37,0.37,0.37}{\textit{#1}}}

\providecommand{\tightlist}{%
  \setlength{\itemsep}{0pt}\setlength{\parskip}{0pt}}\usepackage{longtable,booktabs,array}
\usepackage{calc} % for calculating minipage widths
% Correct order of tables after \paragraph or \subparagraph
\usepackage{etoolbox}
\makeatletter
\patchcmd\longtable{\par}{\if@noskipsec\mbox{}\fi\par}{}{}
\makeatother
% Allow footnotes in longtable head/foot
\IfFileExists{footnotehyper.sty}{\usepackage{footnotehyper}}{\usepackage{footnote}}
\makesavenoteenv{longtable}
\usepackage{graphicx}
\makeatletter
\def\maxwidth{\ifdim\Gin@nat@width>\linewidth\linewidth\else\Gin@nat@width\fi}
\def\maxheight{\ifdim\Gin@nat@height>\textheight\textheight\else\Gin@nat@height\fi}
\makeatother
% Scale images if necessary, so that they will not overflow the page
% margins by default, and it is still possible to overwrite the defaults
% using explicit options in \includegraphics[width, height, ...]{}
\setkeys{Gin}{width=\maxwidth,height=\maxheight,keepaspectratio}
% Set default figure placement to htbp
\makeatletter
\def\fps@figure{htbp}
\makeatother
\newlength{\cslhangindent}
\setlength{\cslhangindent}{1.5em}
\newlength{\csllabelwidth}
\setlength{\csllabelwidth}{3em}
\newlength{\cslentryspacingunit} % times entry-spacing
\setlength{\cslentryspacingunit}{\parskip}
\newenvironment{CSLReferences}[2] % #1 hanging-ident, #2 entry spacing
 {% don't indent paragraphs
  \setlength{\parindent}{0pt}
  % turn on hanging indent if param 1 is 1
  \ifodd #1
  \let\oldpar\par
  \def\par{\hangindent=\cslhangindent\oldpar}
  \fi
  % set entry spacing
  \setlength{\parskip}{#2\cslentryspacingunit}
 }%
 {}
\usepackage{calc}
\newcommand{\CSLBlock}[1]{#1\hfill\break}
\newcommand{\CSLLeftMargin}[1]{\parbox[t]{\csllabelwidth}{#1}}
\newcommand{\CSLRightInline}[1]{\parbox[t]{\linewidth - \csllabelwidth}{#1}\break}
\newcommand{\CSLIndent}[1]{\hspace{\cslhangindent}#1}

\KOMAoption{captions}{tableheading}
\makeatletter
\makeatother
\makeatletter
\@ifpackageloaded{bookmark}{}{\usepackage{bookmark}}
\makeatother
\makeatletter
\@ifpackageloaded{caption}{}{\usepackage{caption}}
\AtBeginDocument{%
\ifdefined\contentsname
  \renewcommand*\contentsname{Table of contents}
\else
  \newcommand\contentsname{Table of contents}
\fi
\ifdefined\listfigurename
  \renewcommand*\listfigurename{List of Figures}
\else
  \newcommand\listfigurename{List of Figures}
\fi
\ifdefined\listtablename
  \renewcommand*\listtablename{List of Tables}
\else
  \newcommand\listtablename{List of Tables}
\fi
\ifdefined\figurename
  \renewcommand*\figurename{Figure}
\else
  \newcommand\figurename{Figure}
\fi
\ifdefined\tablename
  \renewcommand*\tablename{Table}
\else
  \newcommand\tablename{Table}
\fi
}
\@ifpackageloaded{float}{}{\usepackage{float}}
\floatstyle{ruled}
\@ifundefined{c@chapter}{\newfloat{codelisting}{h}{lop}}{\newfloat{codelisting}{h}{lop}[chapter]}
\floatname{codelisting}{Listing}
\newcommand*\listoflistings{\listof{codelisting}{List of Listings}}
\makeatother
\makeatletter
\@ifpackageloaded{caption}{}{\usepackage{caption}}
\@ifpackageloaded{subcaption}{}{\usepackage{subcaption}}
\makeatother
\makeatletter
\@ifpackageloaded{tcolorbox}{}{\usepackage[skins,breakable]{tcolorbox}}
\makeatother
\makeatletter
\@ifundefined{shadecolor}{\definecolor{shadecolor}{rgb}{.97, .97, .97}}
\makeatother
\makeatletter
\makeatother
\makeatletter
\makeatother
\ifLuaTeX
  \usepackage{selnolig}  % disable illegal ligatures
\fi
\IfFileExists{bookmark.sty}{\usepackage{bookmark}}{\usepackage{hyperref}}
\IfFileExists{xurl.sty}{\usepackage{xurl}}{} % add URL line breaks if available
\urlstyle{same} % disable monospaced font for URLs
\hypersetup{
  pdftitle={NCCS Research Guide},
  pdfauthor={Jesse Lecy; Hannah Martin},
  colorlinks=true,
  linkcolor={blue},
  filecolor={Maroon},
  citecolor={Blue},
  urlcolor={Blue},
  pdfcreator={LaTeX via pandoc}}

\title{NCCS Research Guide}
\author{Jesse Lecy \and Hannah Martin}
\date{2023-09-19}

\begin{document}
\maketitle
\ifdefined\Shaded\renewenvironment{Shaded}{\begin{tcolorbox}[interior hidden, borderline west={3pt}{0pt}{shadecolor}, enhanced, sharp corners, breakable, frame hidden, boxrule=0pt]}{\end{tcolorbox}}\fi

\renewcommand*\contentsname{Table of contents}
{
\hypersetup{linkcolor=}
\setcounter{tocdepth}{2}
\tableofcontents
}
\bookmarksetup{startatroot}

\hypertarget{preface}{%
\chapter*{Preface}\label{preface}}
\addcontentsline{toc}{chapter}{Preface}

\markboth{Preface}{Preface}

This is a Quarto book.

To learn more about Quarto books visit
\url{https://quarto.org/docs/books}.

\begin{Shaded}
\begin{Highlighting}[]
\DecValTok{1} \SpecialCharTok{+} \DecValTok{1}
\end{Highlighting}
\end{Shaded}

\begin{verbatim}
[1] 2
\end{verbatim}

\bookmarksetup{startatroot}

\hypertarget{introduction}{%
\chapter{Introduction}\label{introduction}}

This is a book created from markdown and executable code.

See Knuth (1984) for additional discussion of literate programming.

\begin{Shaded}
\begin{Highlighting}[]
\DecValTok{1} \SpecialCharTok{+} \DecValTok{1}
\end{Highlighting}
\end{Shaded}

\begin{verbatim}
[1] 2
\end{verbatim}

\bookmarksetup{startatroot}

\hypertarget{summary}{%
\chapter{Summary}\label{summary}}

In summary, this book has no content whatsoever.

\begin{Shaded}
\begin{Highlighting}[]
\DecValTok{1} \SpecialCharTok{+} \DecValTok{1}
\end{Highlighting}
\end{Shaded}

\begin{verbatim}
[1] 2
\end{verbatim}

\begin{Shaded}
\begin{Highlighting}[]
\NormalTok{dat}\SpecialCharTok{$}\NormalTok{hour12 }\OtherTok{\textless{}{-}} \FunctionTok{format}\NormalTok{( date.vec, }\AttributeTok{format=}\StringTok{"\%l \%p"}\NormalTok{ )}
\FunctionTok{table}\NormalTok{( dat}\SpecialCharTok{$}\NormalTok{hour12 ) }\SpecialCharTok{\%\textgreater{}\%} \FunctionTok{head}\NormalTok{() }\SpecialCharTok{\%\textgreater{}\%} \FunctionTok{pander}\NormalTok{()}

\CommentTok{\# set the levels so they are in the correct order}
\NormalTok{time.levels }\OtherTok{\textless{}{-}}
  \FunctionTok{c}\NormalTok{( }\StringTok{"12 AM"}\NormalTok{, }\StringTok{" 1 AM"}\NormalTok{, }\StringTok{" 2 AM"}\NormalTok{, }\StringTok{" 3 AM"}\NormalTok{, }\StringTok{" 4 AM"}\NormalTok{, }\StringTok{" 5 AM"}\NormalTok{, }
     \StringTok{" 6 AM"}\NormalTok{, }\StringTok{" 7 AM"}\NormalTok{, }\StringTok{" 8 AM"}\NormalTok{, }\StringTok{" 9 AM"}\NormalTok{, }\StringTok{"10 AM"}\NormalTok{, }\StringTok{"11 AM"}\NormalTok{, }
     \StringTok{"12 PM"}\NormalTok{, }\StringTok{" 1 PM"}\NormalTok{, }\StringTok{" 2 PM"}\NormalTok{, }\StringTok{" 3 PM"}\NormalTok{, }\StringTok{" 4 PM"}\NormalTok{, }\StringTok{" 5 PM"}\NormalTok{, }
     \StringTok{" 6 PM"}\NormalTok{, }\StringTok{" 7 PM"}\NormalTok{, }\StringTok{" 8 PM"}\NormalTok{, }\StringTok{" 9 PM"}\NormalTok{, }\StringTok{"10 PM"}\NormalTok{, }\StringTok{"11 PM"}\NormalTok{ )}

\NormalTok{dat}\SpecialCharTok{$}\NormalTok{hour12 }\OtherTok{\textless{}{-}} \FunctionTok{factor}\NormalTok{( dat}\SpecialCharTok{$}\NormalTok{hour12, }\AttributeTok{levels=}\NormalTok{time.levels )}
\FunctionTok{table}\NormalTok{( dat}\SpecialCharTok{$}\NormalTok{hour12 ) }\SpecialCharTok{\%\textgreater{}\%} \FunctionTok{head}\NormalTok{() }\SpecialCharTok{\%\textgreater{}\%} \FunctionTok{pander}\NormalTok{()}
\end{Highlighting}
\end{Shaded}

\begin{Shaded}
\begin{Highlighting}[]
\FunctionTok{qplot}\NormalTok{( }\AttributeTok{data=}\NormalTok{d3, }\AttributeTok{x=}\FunctionTok{as.numeric}\NormalTok{(}\FunctionTok{as.character}\NormalTok{(hour)), }\AttributeTok{y=}\NormalTok{harm ) }\SpecialCharTok{+} 
  \FunctionTok{geom\_line}\NormalTok{( }\AttributeTok{color=}\StringTok{"steelblue"}\NormalTok{, }\AttributeTok{size=}\FloatTok{0.8}\NormalTok{ ) }\SpecialCharTok{+} 
  \FunctionTok{geom\_point}\NormalTok{( }\AttributeTok{color=}\StringTok{"darkblue"}\NormalTok{, }\AttributeTok{size=}\DecValTok{3}\NormalTok{ ) }\SpecialCharTok{+} 
  \FunctionTok{geom\_hline}\NormalTok{( }\AttributeTok{yintercept=}\NormalTok{mean.harm, }\AttributeTok{color=}\StringTok{"black"}\NormalTok{ ) }\SpecialCharTok{+} 
  \FunctionTok{facet\_wrap}\NormalTok{( }\SpecialCharTok{\textasciitilde{}}\NormalTok{ age, }\AttributeTok{ncol=}\DecValTok{4}\NormalTok{ ) }\SpecialCharTok{+} 
  \FunctionTok{xlab}\NormalTok{(}\StringTok{"Time of Day (24hrs)"}\NormalTok{) }\SpecialCharTok{+} \FunctionTok{ylab}\NormalTok{(}\StringTok{"Rate of Harm"}\NormalTok{) }\SpecialCharTok{+}
  \FunctionTok{ggtitle}\NormalTok{(}\StringTok{"Proportion of Accidents Resulting in Harm"}\NormalTok{) }\SpecialCharTok{+}
  \CommentTok{\# theme\_fivethirtyeight() }
  \FunctionTok{theme\_wsj}\NormalTok{( }\AttributeTok{base\_size=}\DecValTok{10}\NormalTok{, }\AttributeTok{color=}\StringTok{"gray"}\NormalTok{ )}
\end{Highlighting}
\end{Shaded}

\bookmarksetup{startatroot}

\hypertarget{summary-1}{%
\chapter{Summary}\label{summary-1}}

In summary, this book has no content whatsoever.

\begin{Shaded}
\begin{Highlighting}[]
\DecValTok{1} \SpecialCharTok{+} \DecValTok{1}
\end{Highlighting}
\end{Shaded}

\begin{verbatim}
[1] 2
\end{verbatim}

\begin{Shaded}
\begin{Highlighting}[]
\NormalTok{dat}\SpecialCharTok{$}\NormalTok{hour12 }\OtherTok{\textless{}{-}} \FunctionTok{format}\NormalTok{( date.vec, }\AttributeTok{format=}\StringTok{"\%l \%p"}\NormalTok{ )}
\FunctionTok{table}\NormalTok{( dat}\SpecialCharTok{$}\NormalTok{hour12 ) }\SpecialCharTok{\%\textgreater{}\%} \FunctionTok{head}\NormalTok{() }\SpecialCharTok{\%\textgreater{}\%} \FunctionTok{pander}\NormalTok{()}

\CommentTok{\# set the levels so they are in the correct order}
\NormalTok{time.levels }\OtherTok{\textless{}{-}}
  \FunctionTok{c}\NormalTok{( }\StringTok{"12 AM"}\NormalTok{, }\StringTok{" 1 AM"}\NormalTok{, }\StringTok{" 2 AM"}\NormalTok{, }\StringTok{" 3 AM"}\NormalTok{, }\StringTok{" 4 AM"}\NormalTok{, }\StringTok{" 5 AM"}\NormalTok{, }
     \StringTok{" 6 AM"}\NormalTok{, }\StringTok{" 7 AM"}\NormalTok{, }\StringTok{" 8 AM"}\NormalTok{, }\StringTok{" 9 AM"}\NormalTok{, }\StringTok{"10 AM"}\NormalTok{, }\StringTok{"11 AM"}\NormalTok{, }
     \StringTok{"12 PM"}\NormalTok{, }\StringTok{" 1 PM"}\NormalTok{, }\StringTok{" 2 PM"}\NormalTok{, }\StringTok{" 3 PM"}\NormalTok{, }\StringTok{" 4 PM"}\NormalTok{, }\StringTok{" 5 PM"}\NormalTok{, }
     \StringTok{" 6 PM"}\NormalTok{, }\StringTok{" 7 PM"}\NormalTok{, }\StringTok{" 8 PM"}\NormalTok{, }\StringTok{" 9 PM"}\NormalTok{, }\StringTok{"10 PM"}\NormalTok{, }\StringTok{"11 PM"}\NormalTok{ )}

\NormalTok{dat}\SpecialCharTok{$}\NormalTok{hour12 }\OtherTok{\textless{}{-}} \FunctionTok{factor}\NormalTok{( dat}\SpecialCharTok{$}\NormalTok{hour12, }\AttributeTok{levels=}\NormalTok{time.levels )}
\FunctionTok{table}\NormalTok{( dat}\SpecialCharTok{$}\NormalTok{hour12 ) }\SpecialCharTok{\%\textgreater{}\%} \FunctionTok{head}\NormalTok{() }\SpecialCharTok{\%\textgreater{}\%} \FunctionTok{pander}\NormalTok{()}
\end{Highlighting}
\end{Shaded}

\begin{Shaded}
\begin{Highlighting}[]
\FunctionTok{qplot}\NormalTok{( }\AttributeTok{data=}\NormalTok{d3, }\AttributeTok{x=}\FunctionTok{as.numeric}\NormalTok{(}\FunctionTok{as.character}\NormalTok{(hour)), }\AttributeTok{y=}\NormalTok{harm ) }\SpecialCharTok{+} 
  \FunctionTok{geom\_line}\NormalTok{( }\AttributeTok{color=}\StringTok{"steelblue"}\NormalTok{, }\AttributeTok{size=}\FloatTok{0.8}\NormalTok{ ) }\SpecialCharTok{+} 
  \FunctionTok{geom\_point}\NormalTok{( }\AttributeTok{color=}\StringTok{"darkblue"}\NormalTok{, }\AttributeTok{size=}\DecValTok{3}\NormalTok{ ) }\SpecialCharTok{+} 
  \FunctionTok{geom\_hline}\NormalTok{( }\AttributeTok{yintercept=}\NormalTok{mean.harm, }\AttributeTok{color=}\StringTok{"black"}\NormalTok{ ) }\SpecialCharTok{+} 
  \FunctionTok{facet\_wrap}\NormalTok{( }\SpecialCharTok{\textasciitilde{}}\NormalTok{ age, }\AttributeTok{ncol=}\DecValTok{4}\NormalTok{ ) }\SpecialCharTok{+} 
  \FunctionTok{xlab}\NormalTok{(}\StringTok{"Time of Day (24hrs)"}\NormalTok{) }\SpecialCharTok{+} \FunctionTok{ylab}\NormalTok{(}\StringTok{"Rate of Harm"}\NormalTok{) }\SpecialCharTok{+}
  \FunctionTok{ggtitle}\NormalTok{(}\StringTok{"Proportion of Accidents Resulting in Harm"}\NormalTok{) }\SpecialCharTok{+}
  \CommentTok{\# theme\_fivethirtyeight() }
  \FunctionTok{theme\_wsj}\NormalTok{( }\AttributeTok{base\_size=}\DecValTok{10}\NormalTok{, }\AttributeTok{color=}\StringTok{"gray"}\NormalTok{ )}
\end{Highlighting}
\end{Shaded}

\bookmarksetup{startatroot}

\hypertarget{summary-2}{%
\chapter{Summary}\label{summary-2}}

In summary, this book has no content whatsoever.

\begin{Shaded}
\begin{Highlighting}[]
\DecValTok{1} \SpecialCharTok{+} \DecValTok{1}
\end{Highlighting}
\end{Shaded}

\begin{verbatim}
[1] 2
\end{verbatim}

\begin{Shaded}
\begin{Highlighting}[]
\NormalTok{dat}\SpecialCharTok{$}\NormalTok{hour12 }\OtherTok{\textless{}{-}} \FunctionTok{format}\NormalTok{( date.vec, }\AttributeTok{format=}\StringTok{"\%l \%p"}\NormalTok{ )}
\FunctionTok{table}\NormalTok{( dat}\SpecialCharTok{$}\NormalTok{hour12 ) }\SpecialCharTok{\%\textgreater{}\%} \FunctionTok{head}\NormalTok{() }\SpecialCharTok{\%\textgreater{}\%} \FunctionTok{pander}\NormalTok{()}

\CommentTok{\# set the levels so they are in the correct order}
\NormalTok{time.levels }\OtherTok{\textless{}{-}}
  \FunctionTok{c}\NormalTok{( }\StringTok{"12 AM"}\NormalTok{, }\StringTok{" 1 AM"}\NormalTok{, }\StringTok{" 2 AM"}\NormalTok{, }\StringTok{" 3 AM"}\NormalTok{, }\StringTok{" 4 AM"}\NormalTok{, }\StringTok{" 5 AM"}\NormalTok{, }
     \StringTok{" 6 AM"}\NormalTok{, }\StringTok{" 7 AM"}\NormalTok{, }\StringTok{" 8 AM"}\NormalTok{, }\StringTok{" 9 AM"}\NormalTok{, }\StringTok{"10 AM"}\NormalTok{, }\StringTok{"11 AM"}\NormalTok{, }
     \StringTok{"12 PM"}\NormalTok{, }\StringTok{" 1 PM"}\NormalTok{, }\StringTok{" 2 PM"}\NormalTok{, }\StringTok{" 3 PM"}\NormalTok{, }\StringTok{" 4 PM"}\NormalTok{, }\StringTok{" 5 PM"}\NormalTok{, }
     \StringTok{" 6 PM"}\NormalTok{, }\StringTok{" 7 PM"}\NormalTok{, }\StringTok{" 8 PM"}\NormalTok{, }\StringTok{" 9 PM"}\NormalTok{, }\StringTok{"10 PM"}\NormalTok{, }\StringTok{"11 PM"}\NormalTok{ )}

\NormalTok{dat}\SpecialCharTok{$}\NormalTok{hour12 }\OtherTok{\textless{}{-}} \FunctionTok{factor}\NormalTok{( dat}\SpecialCharTok{$}\NormalTok{hour12, }\AttributeTok{levels=}\NormalTok{time.levels )}
\FunctionTok{table}\NormalTok{( dat}\SpecialCharTok{$}\NormalTok{hour12 ) }\SpecialCharTok{\%\textgreater{}\%} \FunctionTok{head}\NormalTok{() }\SpecialCharTok{\%\textgreater{}\%} \FunctionTok{pander}\NormalTok{()}
\end{Highlighting}
\end{Shaded}

\begin{Shaded}
\begin{Highlighting}[]
\FunctionTok{qplot}\NormalTok{( }\AttributeTok{data=}\NormalTok{d3, }\AttributeTok{x=}\FunctionTok{as.numeric}\NormalTok{(}\FunctionTok{as.character}\NormalTok{(hour)), }\AttributeTok{y=}\NormalTok{harm ) }\SpecialCharTok{+} 
  \FunctionTok{geom\_line}\NormalTok{( }\AttributeTok{color=}\StringTok{"steelblue"}\NormalTok{, }\AttributeTok{size=}\FloatTok{0.8}\NormalTok{ ) }\SpecialCharTok{+} 
  \FunctionTok{geom\_point}\NormalTok{( }\AttributeTok{color=}\StringTok{"darkblue"}\NormalTok{, }\AttributeTok{size=}\DecValTok{3}\NormalTok{ ) }\SpecialCharTok{+} 
  \FunctionTok{geom\_hline}\NormalTok{( }\AttributeTok{yintercept=}\NormalTok{mean.harm, }\AttributeTok{color=}\StringTok{"black"}\NormalTok{ ) }\SpecialCharTok{+} 
  \FunctionTok{facet\_wrap}\NormalTok{( }\SpecialCharTok{\textasciitilde{}}\NormalTok{ age, }\AttributeTok{ncol=}\DecValTok{4}\NormalTok{ ) }\SpecialCharTok{+} 
  \FunctionTok{xlab}\NormalTok{(}\StringTok{"Time of Day (24hrs)"}\NormalTok{) }\SpecialCharTok{+} \FunctionTok{ylab}\NormalTok{(}\StringTok{"Rate of Harm"}\NormalTok{) }\SpecialCharTok{+}
  \FunctionTok{ggtitle}\NormalTok{(}\StringTok{"Proportion of Accidents Resulting in Harm"}\NormalTok{) }\SpecialCharTok{+}
  \CommentTok{\# theme\_fivethirtyeight() }
  \FunctionTok{theme\_wsj}\NormalTok{( }\AttributeTok{base\_size=}\DecValTok{10}\NormalTok{, }\AttributeTok{color=}\StringTok{"gray"}\NormalTok{ )}
\end{Highlighting}
\end{Shaded}

\bookmarksetup{startatroot}

\hypertarget{summary-3}{%
\chapter{Summary}\label{summary-3}}

In summary, this book has no content whatsoever.

\begin{Shaded}
\begin{Highlighting}[]
\DecValTok{1} \SpecialCharTok{+} \DecValTok{1}
\end{Highlighting}
\end{Shaded}

\begin{verbatim}
[1] 2
\end{verbatim}

\begin{Shaded}
\begin{Highlighting}[]
\NormalTok{dat}\SpecialCharTok{$}\NormalTok{hour12 }\OtherTok{\textless{}{-}} \FunctionTok{format}\NormalTok{( date.vec, }\AttributeTok{format=}\StringTok{"\%l \%p"}\NormalTok{ )}
\FunctionTok{table}\NormalTok{( dat}\SpecialCharTok{$}\NormalTok{hour12 ) }\SpecialCharTok{\%\textgreater{}\%} \FunctionTok{head}\NormalTok{() }\SpecialCharTok{\%\textgreater{}\%} \FunctionTok{pander}\NormalTok{()}

\CommentTok{\# set the levels so they are in the correct order}
\NormalTok{time.levels }\OtherTok{\textless{}{-}}
  \FunctionTok{c}\NormalTok{( }\StringTok{"12 AM"}\NormalTok{, }\StringTok{" 1 AM"}\NormalTok{, }\StringTok{" 2 AM"}\NormalTok{, }\StringTok{" 3 AM"}\NormalTok{, }\StringTok{" 4 AM"}\NormalTok{, }\StringTok{" 5 AM"}\NormalTok{, }
     \StringTok{" 6 AM"}\NormalTok{, }\StringTok{" 7 AM"}\NormalTok{, }\StringTok{" 8 AM"}\NormalTok{, }\StringTok{" 9 AM"}\NormalTok{, }\StringTok{"10 AM"}\NormalTok{, }\StringTok{"11 AM"}\NormalTok{, }
     \StringTok{"12 PM"}\NormalTok{, }\StringTok{" 1 PM"}\NormalTok{, }\StringTok{" 2 PM"}\NormalTok{, }\StringTok{" 3 PM"}\NormalTok{, }\StringTok{" 4 PM"}\NormalTok{, }\StringTok{" 5 PM"}\NormalTok{, }
     \StringTok{" 6 PM"}\NormalTok{, }\StringTok{" 7 PM"}\NormalTok{, }\StringTok{" 8 PM"}\NormalTok{, }\StringTok{" 9 PM"}\NormalTok{, }\StringTok{"10 PM"}\NormalTok{, }\StringTok{"11 PM"}\NormalTok{ )}

\NormalTok{dat}\SpecialCharTok{$}\NormalTok{hour12 }\OtherTok{\textless{}{-}} \FunctionTok{factor}\NormalTok{( dat}\SpecialCharTok{$}\NormalTok{hour12, }\AttributeTok{levels=}\NormalTok{time.levels )}
\FunctionTok{table}\NormalTok{( dat}\SpecialCharTok{$}\NormalTok{hour12 ) }\SpecialCharTok{\%\textgreater{}\%} \FunctionTok{head}\NormalTok{() }\SpecialCharTok{\%\textgreater{}\%} \FunctionTok{pander}\NormalTok{()}
\end{Highlighting}
\end{Shaded}

\begin{Shaded}
\begin{Highlighting}[]
\FunctionTok{qplot}\NormalTok{( }\AttributeTok{data=}\NormalTok{d3, }\AttributeTok{x=}\FunctionTok{as.numeric}\NormalTok{(}\FunctionTok{as.character}\NormalTok{(hour)), }\AttributeTok{y=}\NormalTok{harm ) }\SpecialCharTok{+} 
  \FunctionTok{geom\_line}\NormalTok{( }\AttributeTok{color=}\StringTok{"steelblue"}\NormalTok{, }\AttributeTok{size=}\FloatTok{0.8}\NormalTok{ ) }\SpecialCharTok{+} 
  \FunctionTok{geom\_point}\NormalTok{( }\AttributeTok{color=}\StringTok{"darkblue"}\NormalTok{, }\AttributeTok{size=}\DecValTok{3}\NormalTok{ ) }\SpecialCharTok{+} 
  \FunctionTok{geom\_hline}\NormalTok{( }\AttributeTok{yintercept=}\NormalTok{mean.harm, }\AttributeTok{color=}\StringTok{"black"}\NormalTok{ ) }\SpecialCharTok{+} 
  \FunctionTok{facet\_wrap}\NormalTok{( }\SpecialCharTok{\textasciitilde{}}\NormalTok{ age, }\AttributeTok{ncol=}\DecValTok{4}\NormalTok{ ) }\SpecialCharTok{+} 
  \FunctionTok{xlab}\NormalTok{(}\StringTok{"Time of Day (24hrs)"}\NormalTok{) }\SpecialCharTok{+} \FunctionTok{ylab}\NormalTok{(}\StringTok{"Rate of Harm"}\NormalTok{) }\SpecialCharTok{+}
  \FunctionTok{ggtitle}\NormalTok{(}\StringTok{"Proportion of Accidents Resulting in Harm"}\NormalTok{) }\SpecialCharTok{+}
  \CommentTok{\# theme\_fivethirtyeight() }
  \FunctionTok{theme\_wsj}\NormalTok{( }\AttributeTok{base\_size=}\DecValTok{10}\NormalTok{, }\AttributeTok{color=}\StringTok{"gray"}\NormalTok{ )}
\end{Highlighting}
\end{Shaded}

\bookmarksetup{startatroot}

\hypertarget{summary-4}{%
\chapter{Summary}\label{summary-4}}

In summary, this book has no content whatsoever.

\begin{Shaded}
\begin{Highlighting}[]
\DecValTok{1} \SpecialCharTok{+} \DecValTok{1}
\end{Highlighting}
\end{Shaded}

\begin{verbatim}
[1] 2
\end{verbatim}

\begin{Shaded}
\begin{Highlighting}[]
\NormalTok{dat}\SpecialCharTok{$}\NormalTok{hour12 }\OtherTok{\textless{}{-}} \FunctionTok{format}\NormalTok{( date.vec, }\AttributeTok{format=}\StringTok{"\%l \%p"}\NormalTok{ )}
\FunctionTok{table}\NormalTok{( dat}\SpecialCharTok{$}\NormalTok{hour12 ) }\SpecialCharTok{\%\textgreater{}\%} \FunctionTok{head}\NormalTok{() }\SpecialCharTok{\%\textgreater{}\%} \FunctionTok{pander}\NormalTok{()}

\CommentTok{\# set the levels so they are in the correct order}
\NormalTok{time.levels }\OtherTok{\textless{}{-}}
  \FunctionTok{c}\NormalTok{( }\StringTok{"12 AM"}\NormalTok{, }\StringTok{" 1 AM"}\NormalTok{, }\StringTok{" 2 AM"}\NormalTok{, }\StringTok{" 3 AM"}\NormalTok{, }\StringTok{" 4 AM"}\NormalTok{, }\StringTok{" 5 AM"}\NormalTok{, }
     \StringTok{" 6 AM"}\NormalTok{, }\StringTok{" 7 AM"}\NormalTok{, }\StringTok{" 8 AM"}\NormalTok{, }\StringTok{" 9 AM"}\NormalTok{, }\StringTok{"10 AM"}\NormalTok{, }\StringTok{"11 AM"}\NormalTok{, }
     \StringTok{"12 PM"}\NormalTok{, }\StringTok{" 1 PM"}\NormalTok{, }\StringTok{" 2 PM"}\NormalTok{, }\StringTok{" 3 PM"}\NormalTok{, }\StringTok{" 4 PM"}\NormalTok{, }\StringTok{" 5 PM"}\NormalTok{, }
     \StringTok{" 6 PM"}\NormalTok{, }\StringTok{" 7 PM"}\NormalTok{, }\StringTok{" 8 PM"}\NormalTok{, }\StringTok{" 9 PM"}\NormalTok{, }\StringTok{"10 PM"}\NormalTok{, }\StringTok{"11 PM"}\NormalTok{ )}

\NormalTok{dat}\SpecialCharTok{$}\NormalTok{hour12 }\OtherTok{\textless{}{-}} \FunctionTok{factor}\NormalTok{( dat}\SpecialCharTok{$}\NormalTok{hour12, }\AttributeTok{levels=}\NormalTok{time.levels )}
\FunctionTok{table}\NormalTok{( dat}\SpecialCharTok{$}\NormalTok{hour12 ) }\SpecialCharTok{\%\textgreater{}\%} \FunctionTok{head}\NormalTok{() }\SpecialCharTok{\%\textgreater{}\%} \FunctionTok{pander}\NormalTok{()}
\end{Highlighting}
\end{Shaded}

\begin{Shaded}
\begin{Highlighting}[]
\FunctionTok{qplot}\NormalTok{( }\AttributeTok{data=}\NormalTok{d3, }\AttributeTok{x=}\FunctionTok{as.numeric}\NormalTok{(}\FunctionTok{as.character}\NormalTok{(hour)), }\AttributeTok{y=}\NormalTok{harm ) }\SpecialCharTok{+} 
  \FunctionTok{geom\_line}\NormalTok{( }\AttributeTok{color=}\StringTok{"steelblue"}\NormalTok{, }\AttributeTok{size=}\FloatTok{0.8}\NormalTok{ ) }\SpecialCharTok{+} 
  \FunctionTok{geom\_point}\NormalTok{( }\AttributeTok{color=}\StringTok{"darkblue"}\NormalTok{, }\AttributeTok{size=}\DecValTok{3}\NormalTok{ ) }\SpecialCharTok{+} 
  \FunctionTok{geom\_hline}\NormalTok{( }\AttributeTok{yintercept=}\NormalTok{mean.harm, }\AttributeTok{color=}\StringTok{"black"}\NormalTok{ ) }\SpecialCharTok{+} 
  \FunctionTok{facet\_wrap}\NormalTok{( }\SpecialCharTok{\textasciitilde{}}\NormalTok{ age, }\AttributeTok{ncol=}\DecValTok{4}\NormalTok{ ) }\SpecialCharTok{+} 
  \FunctionTok{xlab}\NormalTok{(}\StringTok{"Time of Day (24hrs)"}\NormalTok{) }\SpecialCharTok{+} \FunctionTok{ylab}\NormalTok{(}\StringTok{"Rate of Harm"}\NormalTok{) }\SpecialCharTok{+}
  \FunctionTok{ggtitle}\NormalTok{(}\StringTok{"Proportion of Accidents Resulting in Harm"}\NormalTok{) }\SpecialCharTok{+}
  \CommentTok{\# theme\_fivethirtyeight() }
  \FunctionTok{theme\_wsj}\NormalTok{( }\AttributeTok{base\_size=}\DecValTok{10}\NormalTok{, }\AttributeTok{color=}\StringTok{"gray"}\NormalTok{ )}
\end{Highlighting}
\end{Shaded}

\bookmarksetup{startatroot}

\hypertarget{foundations-trusts-and-grant-making-organizations}{%
\chapter{Foundations, Trusts, and Grant-Making
Organizations}\label{foundations-trusts-and-grant-making-organizations}}

This chapter explains NCCS's taxonomy for foundations, trusts, and
grant-making organizations:

\begin{longtable}[]{@{}
  >{\raggedright\arraybackslash}p{(\columnwidth - 2\tabcolsep) * \real{0.1831}}
  >{\raggedright\arraybackslash}p{(\columnwidth - 2\tabcolsep) * \real{0.8169}}@{}}
\toprule\noalign{}
\begin{minipage}[b]{\linewidth}\raggedright
Value
\end{minipage} & \begin{minipage}[b]{\linewidth}\raggedright
Organization Type
\end{minipage} \\
\midrule\noalign{}
\endhead
\bottomrule\noalign{}
\endlastfoot
0 & Not a foundation, trust, or grant-making organization \\
1 & 501(c)(3) private foundation, trust, or grant-making organization:
private operating foundation (IRS status) \& corporate foundation (NTEE
code) \\
2 & 501(c)(3) private foundation, trust, or grant-making organization:
private operating foundation (IRS status) \& private independent
foundation (NTEE code) \\
3 & 501(c)(3) private foundation, trust, or grant-making organization:
private operating foundation (IRS status) \& private operating
foundation (NTEE code) \\
4 & 501(c)(3) private foundation, trust, or grant-making organization:
private operating foundation (IRS status), other (NTEE code) \\
5 & 501(c)(3) private foundation, trust, or grant-making organization:
exempt operating foundation (IRS status) \& corporate foundation (NTEE
code) \\
6 & 501(c)(3) private foundation, trust, or grant-making organization:
exempt operating foundation (IRS status) \& private independent
foundation (NTEE code) \\
7 & 501(c)(3) private foundation, trust, or grant-making organization:
exempt operating foundation (IRS status) \& private operating foundation
(NTEE code) \\
8 & 501(c)(3) private foundation, trust, or grant-making organization:
exempt operating foundation (IRS status), other (NTEE code) \\
9 & 501(c)(3) private foundation, trust, or grant-making organization:
grant-making (private nonoperating) foundation (IRS status) \& corporate
foundation (NTEE code) \\
10 & 501(c)(3) private foundation, trust, or grant-making organization:
grant-making (private nonoperating) foundation (IRS status) \& private
independent foundation (NTEE code) \\
11 & 501(c)(3) private foundation, trust, or grant-making organization:
grant-making (private nonoperating) foundation (IRS status) \& private
operating foundation (NTEE code) \\
12 & 501(c)(3) private foundation, trust, or grant-making organization:
grant-making (private nonoperating) foundation (IRS status), other (NTEE
code) \\
13 & 501(c)(3) public foundation, trust, or grant-making organization:
community foundation \\
14 & 501(c)(3) public foundation, trust, or grant-making organization:
supporting organization - single organization support \\
15 & 501(c)(3) public foundation, trust, or grant-making organization:
supporting organization - multiple organization support \\
16 & 501(c)(3) public foundation, trust, or grant-making organization:
other public foundation \\
17 & Other community foundation \\
18 & Other supporting organization - single organization support \\
19 & Other supporting organization - multiple organization support \\
20 & Other public foundation \\
21 & 501(c)(17) trust providing for the payment of supplemental
unemployment compensation benefits \\
22 & 501(c)(18) employee funded pension trust \\
23 & 501(c)(21) black lung benefit trust \\
24 & 4947(a)(1) charitable trust treated as a private foundation \\
25 & 4947(a)(1) charitable trust not treated as a private foundation \\
26 & 4947(a)(2) split-interest charitable trust \\
\end{longtable}

Each type of organization is described in detail below.

\hypertarget{c3-foundations-trusts-and-grant-making-organizations}{%
\section{501(c)(3) foundations, trusts, and grant-making
organizations}\label{c3-foundations-trusts-and-grant-making-organizations}}

To qualify as a 501(c)(3) organization, an organization must exist to
advance one of the following exempt purposes:

\begin{itemize}
\item
  Charitable, which includes:

  \begin{itemize}
  \item
    ``Relief of the poor, the distressed, or the underprivileged;
  \item
    Advancement of religion;
  \item
    Advancement of education or science;
  \item
    Erection or maintenance of public buildings, monuments, or works;
  \item
    Lessening the burdens of government;
  \item
    Lessening neighborhood tensions;
  \item
    Eliminating prejudice and discrimination;
  \item
    Defending human and civil rights secured by law; and
  \item
    Combating community deterioration and juvenile delinquency.''
  \end{itemize}
\item
  Religious
\item
  Educational
\item
  Scientific
\item
  Literary
\item
  Testing for public safety
\item
  Fostering national or international amateur sports competition
\item
  Prevention of cruelty to children or animals\footnote{({``Exempt
    Purposes -- Internal Revenue Code Section 501(c)(3),''} n.d.)}
\end{itemize}

501(c)(3) organizations indicate which exempt purpose(s) they advance
when they file Form 1023 or 1023-EZ to apply for recognition of
exemption from federal income tax (aka ``tax-exempt status'') (Service
2014, 2020).

501(c)(3) organizations are either private foundations or public
charities.

\hypertarget{c3-private-foundations}{%
\subsection{501(c)(3) private
foundations}\label{c3-private-foundations}}

The IRS distinguishes between three types of 501(c)(3) private
foundations: private operating foundations, exempt operating
foundations, and grant-making (private nonoperating) foundations.

501(c)(3) private foundations indicate whether they are a private
operating foundation when they file Form 1023 (Service 2020). This
option is not available to Form 1023-EZ filers (Service 2014). Later,
private operating foundations that want recognition of exempt private
operating foundation status must file Form 8940, Request for
Miscellaneous Determination, to obtain a determination
letter.\footnote{({``Definition of Exempt Operating Foundation,''} n.d.)}

501(c)(3) private foundations must file an annual Form
990-PF.\footnote{({``Instructions for Form 990-PF (2022),''} n.d.)}

\hypertarget{c3-private-foundation-private-operating-foundation-irs-status-corporate-foundation-ntee-code}{%
\subsubsection{501(c)(3) private foundation: private operating
foundation (IRS status) \& corporate foundation (NTEE
code)}\label{c3-private-foundation-private-operating-foundation-irs-status-corporate-foundation-ntee-code}}

Private operating foundations (IRS status) use most of their resources
to actively conduct their exempt activities.\footnote{({``Private
  Operating Foundations,''} n.d.)}

Corporate foundations (NTEE code) are ``private foundations whose grant
funds are derived primarily from the contributions of a profit-making
business organization.''\footnote{({``IRS Activity Codes,''} n.d.)}

We can identify 501(c)(3) private operating foundations (IRS status)
that are also corporate foundations (NTEE code) using FNDNCD (reason for
501(c)(3) status) and NTEECC (NTEECC primary purpose). Values of 3 for
FNDNCD indicate that an organization is a 501(c)(3) private operating
foundation (IRS status), and values of T21 for NTEECC indicate that an
organization is a corporate foundation (NTEE code).

\hypertarget{c3-private-foundation-private-operating-foundation-irs-status-private-independent-foundation-ntee-code}{%
\subsubsection{501(c)(3) private foundation: private operating
foundation (IRS status) \& private independent foundation (NTEE
code)}\label{c3-private-foundation-private-operating-foundation-irs-status-private-independent-foundation-ntee-code}}

Private operating foundations (IRS status) use most of their resources
to actively conduct their exempt activities.\footnote{({``Private
  Operating Foundations,''} n.d.)}

Private independent foundations (NTEE code) are ``private foundations
that make grants based on charitable endowments. Because of their
endowments, they are focused primarily on grantmaking and generally do
not actively raise funds or seek public financial support. These are the
most common type of private foundation They are generally endowed,
usually from a single individual or family. Private foundations are
considered family foundations if relatives or the original donor are
still active on the board of trustees or in the operation of the
foundation.''\footnote{({``IRS Activity Codes,''} n.d.)}

We can identify 501(c)(3) private operating foundations (IRS status)
that are also private independent foundations (NTEE code) using FNDNCD
(reason for 501(c)(3) status) and NTEECC (NTEECC primary purpose).
Values of 3 for FNDNCD indicate that an organization is a 501(c)(3)
private operating foundation (IRS status), and values of T22 for NTEECC
indicate that an organization is a private independent foundation (NTEE
code).

\hypertarget{c3-private-foundation-private-operating-foundation-irs-status-private-operating-foundation-ntee-code}{%
\subsubsection{501(c)(3) private foundation: private operating
foundation (IRS status) \& private operating foundation (NTEE
code)}\label{c3-private-foundation-private-operating-foundation-irs-status-private-operating-foundation-ntee-code}}

Private operating foundations (IRS status) use most of their resources
to actively conduct their exempt activities.\footnote{({``Private
  Operating Foundations,''} n.d.)}

Private operating foundations (NTEE code) are ``private foundations that
use a bulk of their resources to provide charitable services or run
charitable programs of their own. They make few, if any, grants to
outside organizations and, like private independent foundations, they
generally do not raise funds from the public.''\footnote{({``IRS
  Activity Codes,''} n.d.)}

We can identify 501(c)(3) private operating foundations (IRS status)
that are also private operating foundations (NTEE code) using FNDNCD
(reason for 501(c)(3) status) and NTEECC (NTEECC primary purpose).
Values of 3 for FNDNCD indicate that an organization is a 501(c)(3)
private operating foundation (IRS status), and values of T23 for NTEECC
indicate that an organization is a private operating foundation (NTEE
code).

\hypertarget{c3-private-foundation-private-operating-foundation-irs-status-other-ntee-code}{%
\subsubsection{501(c)(3) private foundation: private operating
foundation (IRS status), other (NTEE
code)}\label{c3-private-foundation-private-operating-foundation-irs-status-other-ntee-code}}

Private operating foundations (IRS status) use most of their resources
to actively conduct their exempt activities.\footnote{({``Private
  Operating Foundations,''} n.d.)}

We can identify 501(c)(3) private operating foundations (IRS status)
that are not corporate foundations, private independent foundations, or
private operating foundations (NTEE codes) using FNDNCD (reason for
501(c)(3) status) and NTEECC (NTEECC primary purpose). Values of 3 for
FNDNCD indicate that an organization is a 501(c)(3) private operating
foundation (IRS status), and values of anything other than T21, T22, and
T23 for NTEECC indicate that an organization is not a corporate
foundation, private independent foundation, or private operating
foundation (NTEE codes).

\hypertarget{c3-private-foundation-exempt-operating-foundation-irs-status-corporate-foundation-ntee-code}{%
\subsubsection{501(c)(3) private foundation: exempt operating foundation
(IRS status) \& corporate foundation (NTEE
code)}\label{c3-private-foundation-exempt-operating-foundation-irs-status-corporate-foundation-ntee-code}}

Exempt operating foundations (IRS status) are private operating
foundations that have been publicly supported for at least 10 years,
have governing bodies with less than 25\% disqualified individuals and
that broadly represent the general public, and have no disqualified
individuals as officers.\footnote{({``Exempt Operating Foundations,''}
  n.d.)} ``Disqualified individuals'' in this case refers to substantial
contributors to the foundation; owners of more than 20\% of the total
combined voting power of a corporation, the profits interest of a
partnership, or the beneficial interest of a trust or unincorporated
enterprise (if these entities are substantial contributors to the
foundation); and family members of any individuals previously
described.\footnote{({``'Disqualified Individual' -- Exempt Operating
  Foundation,''} n.d.)}

Corporate foundations (NTEE code) are ``private foundations whose grant
funds are derived primarily from the contributions of a profit-making
business organization.''\footnote{({``IRS Activity Codes,''} n.d.)}

We can identify 501(c)(3) exempt operating foundations (IRS status) that
are also corporate foundations (NTEE code) using FNDNCD (reason for
501(c)(3) status) and NTEECC (NTEECC primary purpose). Values of 2 for
FNDNCD indicate that an organization is a 501(c)(3) exempt operating
foundation (IRS status), and values of T21 for NTEECC indicate that an
organization is a corporate foundation (NTEE code).

\hypertarget{c3-private-foundation-exempt-operating-foundation-irs-status-private-independent-foundation-ntee-code-ntee-code}{%
\subsubsection{501(c)(3) private foundation: exempt operating foundation
(IRS status) \& private independent foundation (NTEE code) (NTEE
code)}\label{c3-private-foundation-exempt-operating-foundation-irs-status-private-independent-foundation-ntee-code-ntee-code}}

Exempt operating foundations (IRS status) are private operating
foundations that have been publicly supported for at least 10 years,
have governing bodies with less than 25\% disqualified individuals and
that broadly represent the general public, and have no disqualified
individuals as officers.\footnote{({``Exempt Operating Foundations,''}
  n.d.)} ``Disqualified individuals'' in this case refers to substantial
contributors to the foundation; owners of more than 20\% of the total
combined voting power of a corporation, the profits interest of a
partnership, or the beneficial interest of a trust or unincorporated
enterprise (if these entities are substantial contributors to the
foundation); and family members of any individuals previously
described.\footnote{({``'Disqualified Individual' -- Exempt Operating
  Foundation,''} n.d.)}

Private independent foundations (NTEE code) are ``private foundations
that make grants based on charitable endowments. Because of their
endowments, they are focused primarily on grantmaking and generally do
not actively raise funds or seek public financial support. These are the
most common type of private foundation They are generally endowed,
usually from a single individual or family. Private foundations are
considered family foundations if relatives or the original donor are
still active on the board of trustees or in the operation of the
foundation.''\footnote{({``IRS Activity Codes,''} n.d.)}

We can identify 501(c)(3) exempt operating foundations (IRS status) that
are also private independent foundations (NTEE code) using FNDNCD
(reason for 501(c)(3) status) and NTEECC (NTEECC primary purpose).
Values of 2 for FNDNCD indicate that an organization is a 501(c)(3)
exempt operating foundation (IRS status), and values of T22 for NTEECC
indicate that an organization is a private independent foundation (NTEE
code).

\hypertarget{c3-private-foundation-exempt-operating-foundation-irs-status-private-operating-foundation-ntee-code}{%
\subsubsection{501(c)(3) private foundation: exempt operating foundation
(IRS status) \& private operating foundation (NTEE
code)}\label{c3-private-foundation-exempt-operating-foundation-irs-status-private-operating-foundation-ntee-code}}

Exempt operating foundations (IRS status) are private operating
foundations that have been publicly supported for at least 10 years,
have governing bodies with less than 25\% disqualified individuals and
that broadly represent the general public, and have no disqualified
individuals as officers.\footnote{({``Exempt Operating Foundations,''}
  n.d.)} ``Disqualified individuals'' in this case refers to substantial
contributors to the foundation; owners of more than 20\% of the total
combined voting power of a corporation, the profits interest of a
partnership, or the beneficial interest of a trust or unincorporated
enterprise (if these entities are substantial contributors to the
foundation); and family members of any individuals previously
described.\footnote{({``'Disqualified Individual' -- Exempt Operating
  Foundation,''} n.d.)}

Private operating foundations (NTEE code) are ``private foundations that
use a bulk of their resources to provide charitable services or run
charitable programs of their own. They make few, if any, grants to
outside organizations and, like private independent foundations, they
generally do not raise funds from the public.''\footnote{({``IRS
  Activity Codes,''} n.d.)}

We can identify 501(c)(3) exempt operating foundations (IRS status) that
are also private operating foundations (NTEE code) using FNDNCD (reason
for 501(c)(3) status) and NTEECC (NTEECC primary purpose). Values of 2
for FNDNCD indicate that an organization is a 501(c)(3) exempt operating
foundation (IRS status), and values of T23 for NTEECC indicate that an
organization is a private operating foundation (NTEE code).

\hypertarget{c3-private-foundation-exempt-operating-foundation-irs-status-other-ntee-code}{%
\subsubsection{501(c)(3) private foundation: exempt operating foundation
(IRS status), other (NTEE
code)}\label{c3-private-foundation-exempt-operating-foundation-irs-status-other-ntee-code}}

Exempt operating foundations (IRS status) are private operating
foundations that have been publicly supported for at least 10 years,
have governing bodies with less than 25\% disqualified individuals and
that broadly represent the general public, and have no disqualified
individuals as officers.\footnote{({``Exempt Operating Foundations,''}
  n.d.)} ``Disqualified individuals'' in this case refers to substantial
contributors to the foundation; owners of more than 20\% of the total
combined voting power of a corporation, the profits interest of a
partnership, or the beneficial interest of a trust or unincorporated
enterprise (if these entities are substantial contributors to the
foundation); and family members of any individuals previously
described.\footnote{({``'Disqualified Individual' -- Exempt Operating
  Foundation,''} n.d.)}

We can identify 501(c)(3) exempt operating foundations (IRS status) that
are not corporate foundations, private independent foundations, or
private operating foundations (NTEE codes) using FNDNCD (reason for
501(c)(3) status) and NTEECC (NTEECC primary purpose). Values of 2 for
FNDNCD indicate that an organization is a 501(c)(3) exempt operating
foundation (IRS status), and values of anything other than T21, T22, and
T23 for NTEECC indicate that an organization is not a corporate
foundation, private independent foundation, or private operating
foundation (NTEE codes).

\hypertarget{c3-private-foundation-grant-making-private-nonoperating-foundation-irs-status-corporate-foundation-ntee-code}{%
\subsubsection{501(c)(3) private foundation: grant-making (private
nonoperating) foundation (IRS status) \& corporate foundation (NTEE
code)}\label{c3-private-foundation-grant-making-private-nonoperating-foundation-irs-status-corporate-foundation-ntee-code}}

Grant-making (private nonoperating) foundations (IRS status) are all
other private foundations.\footnote{({``Grant-Making Foundations,''}
  n.d.)}

Private operating foundations have the following advantages over
grant-making (private nonoperating) foundations:

\begin{itemize}
\tightlist
\item
  They are exempt from the excise tax on failure to distribute income.
\item
  Donors can deduct contributions to these foundations to the extent of
  50\% of their adjusted gross income, instead of 30\%.
\item
  Private foundations can make qualifying distributions to them, as long
  as they are not controlled by said private foundation.\footnote{({``Private
    Operating Foundations,''} n.d.)}
\end{itemize}

In addition, \emph{exempt} operating foundations have the following
advantages over grant-making (private nonoperating) foundations:

\begin{itemize}
\tightlist
\item
  They are exempt from taxes on net investment income.
\item
  Private foundations can make grants to them without following
  expenditure responsibility requirements.\footnote{({``Exempt Operating
    Foundations,''} n.d.)}
\end{itemize}

Corporate foundations (NTEE code) are ``private foundations whose grant
funds are derived primarily from the contributions of a profit-making
business organization.''\footnote{({``IRS Activity Codes,''} n.d.)}

We can identify 501(c)(3) grant-making (private nonoperating)
foundations (IRS status) that are also corporate foundations (NTEE code)
using FNDNCD (reason for 501(c)(3) status) and NTEECC (NTEECC primary
purpose). Values of 4 indicate that an organization is a 501(c)(3)
grant-making (private nonoperating) foundation (IRS status), and values
of T21 for NTEECC indicate that an organization is a corporate
foundation (NTEE code).

\hypertarget{c3-private-foundation-grant-making-private-nonoperating-foundation-irs-status-private-independent-foundation-ntee-code}{%
\subsubsection{501(c)(3) private foundation: grant-making (private
nonoperating) foundation (IRS status) \& private independent foundation
(NTEE
code)}\label{c3-private-foundation-grant-making-private-nonoperating-foundation-irs-status-private-independent-foundation-ntee-code}}

Grant-making (private nonoperating) foundations (IRS status) are all
other private foundations.\footnote{({``Grant-Making Foundations,''}
  n.d.)}

Private operating foundations have the following advantages over
grant-making (private nonoperating) foundations:

\begin{itemize}
\tightlist
\item
  They are exempt from the excise tax on failure to distribute income.
\item
  Donors can deduct contributions to these foundations to the extent of
  50\% of their adjusted gross income, instead of 30\%.
\item
  Private foundations can make qualifying distributions to them, as long
  as they are not controlled by said private foundation.\footnote{({``Private
    Operating Foundations,''} n.d.)}
\end{itemize}

In addition, \emph{exempt} operating foundations have the following
advantages over grant-making (private nonoperating) foundations:

\begin{itemize}
\tightlist
\item
  They are exempt from taxes on net investment income.
\item
  Private foundations can make grants to them without following
  expenditure responsibility requirements.\footnote{({``Exempt Operating
    Foundations,''} n.d.)}
\end{itemize}

Private independent foundations (NTEE code) are ``private foundations
that make grants based on charitable endowments. Because of their
endowments, they are focused primarily on grantmaking and generally do
not actively raise funds or seek public financial support. These are the
most common type of private foundation They are generally endowed,
usually from a single individual or family. Private foundations are
considered family foundations if relatives or the original donor are
still active on the board of trustees or in the operation of the
foundation.''\footnote{({``IRS Activity Codes,''} n.d.)}

We can identify 501(c)(3) grant-making (private nonoperating)
foundations (IRS status) that are also private independent foundations
(NTEE code) using FNDNCD (reason for 501(c)(3) status) and NTEECC
(NTEECC primary purpose). Values of 4 indicate that an organization is a
501(c)(3) grant-making (private nonoperating) foundation (IRS status),
and values of T22 for NTEECC indicate that an organization is a private
independent foundation (NTEE code).

\hypertarget{c3-private-foundation-grant-making-private-nonoperating-foundation-irs-status-private-operating-foundation-ntee-code}{%
\subsubsection{501(c)(3) private foundation: grant-making (private
nonoperating) foundation (IRS status) \& private operating foundation
(NTEE
code)}\label{c3-private-foundation-grant-making-private-nonoperating-foundation-irs-status-private-operating-foundation-ntee-code}}

Grant-making (private nonoperating) foundations (IRS status) are all
other private foundations.\footnote{({``Grant-Making Foundations,''}
  n.d.)}

Private operating foundations have the following advantages over
grant-making (private nonoperating) foundations:

\begin{itemize}
\tightlist
\item
  They are exempt from the excise tax on failure to distribute income.
\item
  Donors can deduct contributions to these foundations to the extent of
  50\% of their adjusted gross income, instead of 30\%.
\item
  Private foundations can make qualifying distributions to them, as long
  as they are not controlled by said private foundation.\footnote{({``Private
    Operating Foundations,''} n.d.)}
\end{itemize}

In addition, \emph{exempt} operating foundations have the following
advantages over grant-making (private nonoperating) foundations:

\begin{itemize}
\tightlist
\item
  They are exempt from taxes on net investment income.
\item
  Private foundations can make grants to them without following
  expenditure responsibility requirements.\footnote{({``Exempt Operating
    Foundations,''} n.d.)}
\end{itemize}

Private operating foundations (NTEE code) are ``private foundations that
use a bulk of their resources to provide charitable services or run
charitable programs of their own. They make few, if any, grants to
outside organizations and, like private independent foundations, they
generally do not raise funds from the public.''\footnote{({``IRS
  Activity Codes,''} n.d.)}

We can identify 501(c)(3) grant-making (private nonoperating)
foundations (IRS status) that are also private operating foundations
(NTEE code) using FNDNCD (reason for 501(c)(3) status) and NTEECC
(NTEECC primary purpose). Values of 4 indicate that an organization is a
501(c)(3) grant-making (private nonoperating) foundation (IRS status),
and values of T23 for NTEECC indicate that an organization is a private
operating foundation (NTEE code).

\hypertarget{c3-private-foundation-grant-making-private-nonoperating-foundation-irs-status-other-ntee-code}{%
\subsubsection{501(c)(3) private foundation: grant-making (private
nonoperating) foundation (IRS status), other (NTEE
code)}\label{c3-private-foundation-grant-making-private-nonoperating-foundation-irs-status-other-ntee-code}}

Grant-making (private nonoperating) foundations (IRS status) are all
other private foundations.\footnote{({``Grant-Making Foundations,''}
  n.d.)}

Private operating foundations have the following advantages over
grant-making (private nonoperating) foundations:

\begin{itemize}
\tightlist
\item
  They are exempt from the excise tax on failure to distribute income.
\item
  Donors can deduct contributions to these foundations to the extent of
  50\% of their adjusted gross income, instead of 30\%.
\item
  Private foundations can make qualifying distributions to them, as long
  as they are not controlled by said private foundation.\footnote{({``Private
    Operating Foundations,''} n.d.)}
\end{itemize}

In addition, \emph{exempt} operating foundations have the following
advantages over grant-making (private nonoperating) foundations:

\begin{itemize}
\tightlist
\item
  They are exempt from taxes on net investment income.
\item
  Private foundations can make grants to them without following
  expenditure responsibility requirements.\footnote{({``Exempt Operating
    Foundations,''} n.d.)}
\end{itemize}

We can identify 501(c)(3) grant-making (private nonoperating)
foundations (IRS status) that are not corporate foundations, private
independent foundations, or private operating foundations (NTEE codes)
using FNDNCD (reason for 501(c)(3) status) and NTEECC (NTEECC primary
purpose). Values of 4 indicate that an organization is a 501(c)(3)
grant-making (private nonoperating) foundation (IRS status), and values
of anything other than T21, T22, and T23 for NTEECC indicate that an
organization is not a corporate foundation, private independent
foundation, or private operating foundation (NTEE codes).

\hypertarget{c3-public-foundations-trusts-and-grant-making-organizations}{%
\subsection{501(c)(3) public foundations, trusts, and grant-making
organizations}\label{c3-public-foundations-trusts-and-grant-making-organizations}}

To qualify as a 501(c)(3) public charity, rather than a 501(c)(3)
private foundation, an organization must be one of the following:

\begin{itemize}
\item
  A church, convention of churches, or association of churches
\item
  A school
\item
  A hospital or cooperative hospital service organization
\item
  A medical research organization operated in conjunction with a
  hospital
\item
  An organization operated for the benefit of a college or university
  that is owned or operated by a governmental unit
\item
  A federal, state, or local government or governmental unit
\item
  An organization that normally receives a substantial part of its
  support from a governmental unit or from the general public
\item
  A community trust
\item
  An agricultural research organization operated in conjunction with a
  land-grant college or university or a non-land-grant college of
  agriculture
\item
  An organization that normally receives more than 33 and 1/3\% of its
  support from contributions, membership fees, and gross receipts from
  activities related to its exempt functions and receives no more than
  33 and 1/3 percent of its support from gross investment income and
  unrelated business taxable income from businesses acquired by the
  organization after June 30, 1975
\item
  An organization organized and operated exclusively for testing for
  public safety
\item
  A 509(a)(3) supporting organization, including the following types:

  \begin{itemize}
  \item
    Type I -- those operated, supervised, or controlled by the supported
    organization(s) by giving the supported organization(s) the power to
    regularly appoint or elect a majority of the directors or trustees
    of the supporting organization
  \item
    Type II -- those supervised or controlled in connection with the
    supported organization(s) by having control or management of the
    supporting organization vested in the same persons that control or
    manage the supported organization(s)
  \item
    Type III functionally integrated -- those operated in connection
    with, and functionally integrated with, the supported
    organization(s)
  \item
    Type III non-functionally integrated -- those operated in connection
    with the supported organization(s) that are not functionally
    integrated (Service 2022)
  \end{itemize}
\end{itemize}

501(c)(3) organizations indicate whether they are a public charity or a
private foundation when they file Form 1023 or 1023-EZ, and if they are
a public charity, they must select the reason they are not a private
foundation (Service 2014, 2020). 501(c)(3) public charities also select
the reason they are not a private foundation when they file Schedule A
(Form 990) every year (Service 2022). Private foundations that later
want to be reclassified as public charities must terminate their private
foundation status and apply for public charity status.\footnote{({``Instructions
  for Form 8940 (04/2023),''} n.d.)}

The IRS imposes several taxes, restrictions, and requirements on private
foundations that it does not place on public charities:

\begin{itemize}
\item
  Private foundations must pay excise taxes on their net investment
  income.
\item
  Private foundations must pay excise taxes on acts of self-dealing with
  disqualified individuals.
\item
  Private foundations must distribute a portion of their income annually
  for charitable purposes.
\item
  Private foundations must pay excise taxes on any excess holdings above
  20\% that it and all of its disqualified individuals have in the
  voting stock of a business enterprise.
\item
  Private foundations must pay excise taxes on jeopardizing investments
  (i.e., those that jeopardize the carrying out of exempt purposes).
\item
  Private foundations must pay excise taxes on expenditures that do not
  further exempt purposes, such as lobbying.\footnote{({``Private
    Foundations,''} n.d.)}
\end{itemize}

Most 501(c)(3) public charities must file an annual Form 990-N, 990-EZ,
or 990.\footnote{({``Annual Exempt Organization Return: Who Must
  File,''} n.d.)}

\hypertarget{c3-public-foundation-trust-or-grant-making-organization-community-foundation}{%
\subsubsection{501(c)(3) public foundation, trust, or grant-making
organization: community
foundation}\label{c3-public-foundation-trust-or-grant-making-organization-community-foundation}}

Community foundations are ``organizations that make grants for
charitable purposes in a specific community or region. The funds
available to a community foundation are usually derived from many donors
and held in an endowment that is independently administered; income
earned by the endowment is then used to make grants.''\footnote{({``IRS
  Activity Codes,''} n.d.)}

We can identify 501(c)(3) community foundations using SUBSECCD
(subsection code) and NTEECC (NTEECC primary purpose). Values of 3 for
SUBSECCD indicate that an organization is a 501(c)(3) organization, and
values of T31 for NTEECC indicate that an organization is a community
foundation.

\hypertarget{c3-public-foundation-trust-or-grant-making-organization-supporting-organization---single-organization-support}{%
\subsubsection{501(c)(3) public foundation, trust, or grant-making
organization: supporting organization - single organization
support}\label{c3-public-foundation-trust-or-grant-making-organization-supporting-organization---single-organization-support}}

Supporting organizations that provide support to a single organization
are ``organizations existing as a support and fund-raising entity for a
single institution.''\footnote{({``IRS Activity Codes,''} n.d.)}

We can identify 501(c)(3) supporting organizations that provide support
to a single organization using SUBSECCD (subsection code) and NTEECC
(NTEECC primary purpose). Values of 3 for SUBSECCD indicate that an
organization is a 501(c)(3) organization, and values of A11, B11, C11,
D11, E11, F11, G11, H11, I11, J11, K11, L11, M11, N11, O11, P11, Q11,
R11, S11, T11, U11, V11, W11, X11, and Y11 for NTEECC indicate that an
organization is a supporting organization that provides support to a
single organization.

\hypertarget{c3-public-foundation-trust-or-grant-making-organization-supporting-organization---multiple-organization-support}{%
\subsubsection{501(c)(3) public foundation, trust, or grant-making
organization: supporting organization - multiple organization
support}\label{c3-public-foundation-trust-or-grant-making-organization-supporting-organization---multiple-organization-support}}

Supporting organizations that provide support to multiple organizations
are ``organizations that raise and distribute funds for multiple
organizations.''\footnote{({``IRS Activity Codes,''} n.d.)}

We can identify 501(c)(3) supporting organizations that provide support
to multiple organizations using SUBSECCD (subsection code) and NTEECC
(NTEECC primary purpose). Values of 3 for SUBSECCD indicate that an
organization is a 501(c)(3) organization, and values of A12, B12, C12,
D12, E12, F12, G12, H12, I12, J12, K12, L12, M12, N12, O12, P12, Q12,
R12, S12, T12, U12, V12, W12, X12, and Y12 for NTEECC indicate that an
organization is a supporting organization that provides support to
multiple organizations.

\hypertarget{c3-public-foundation-trust-or-grant-making-organization-other-public-foundation}{%
\subsubsection{501(c)(3) public foundation, trust, or grant-making
organization: other public
foundation}\label{c3-public-foundation-trust-or-grant-making-organization-other-public-foundation}}

Public foundations are ``organizations that derive their funding or
support primarily from the general public in carrying our their social,
educational, religious or other charitable activities serving the common
welfare. Although public foundations may provide direct charitable
services to the public as other nonprofits do, their primary focus is on
grantmaking.''\footnote{({``IRS Activity Codes,''} n.d.)}

We can identify other 501(c)(3) public foundations using SUBSECCD
(subsection code) and NTEECC (NTEECC primary purpose). Values of 3 for
SUBSECCD indicate that an organization is a 501(c)(3) organization, and
values of T30 for NTEECC indicate that an organization is a public
foundation.

\hypertarget{other-community-foundation}{%
\subsection{Other community
foundation}\label{other-community-foundation}}

Community foundations are ``organizations that make grants for
charitable purposes in a specific community or region. The funds
available to a community foundation are usually derived from many donors
and held in an endowment that is independently administered; income
earned by the endowment is then used to make grants.''\footnote{({``IRS
  Activity Codes,''} n.d.)}

We can identify non-501(c)(3) community foundations using SUBSECCD
(subsection code) and NTEECC (NTEECC primary purpose). Values of
anything other than 3 for SUBSECCD indicate that an organization is a
non-501(c)(3) organization, and values of T31 for NTEECC indicate that
an organization is a community foundation.

\hypertarget{other-supporting-organization---single-organization-support}{%
\subsection{Other supporting organization - single organization
support}\label{other-supporting-organization---single-organization-support}}

Supporting organizations that provide support to a single organization
are ``organizations existing as a support and fund-raising entity for a
single institution.''\footnote{({``IRS Activity Codes,''} n.d.)}

We can identify non-501(c)(3) supporting organizations that provide
support to a single organization using SUBSECCD (subsection code) and
NTEECC (NTEECC primary purpose). Values of anything other than 3 for
SUBSECCD indicate that an organization is a non-501(c)(3) organization,
and values of A11, B11, C11, D11, E11, F11, G11, H11, I11, J11, K11,
L11, M11, N11, O11, P11, Q11, R11, S11, T11, U11, V11, W11, X11, and Y11
for NTEECC indicate that an organization is a supporting organization
that provides support to a single organization.

\hypertarget{other-supporting-organization---multiple-organization-support}{%
\subsection{Other supporting organization - multiple organization
support}\label{other-supporting-organization---multiple-organization-support}}

Supporting organizations that provide support to multiple organizations
are ``organizations that raise and distribute funds for multiple
organizations.''\footnote{({``IRS Activity Codes,''} n.d.)}

We can identify non-501(c)(3) supporting organizations that provide
support to multiple organizations using SUBSECCD (subsection code) and
NTEECC (NTEECC primary purpose). Values of anything other than 3 for
SUBSECCD indicate that an organization is a non-501(c)(3) organization,
and values of A12, B12, C12, D12, E12, F12, G12, H12, I12, J12, K12,
L12, M12, N12, O12, P12, Q12, R12, S12, T12, U12, V12, W12, X12, and Y12
for NTEECC indicate that an organization is a supporting organization
that provides support to multiple organizations.

\hypertarget{other-public-foundation}{%
\subsection{Other public foundation}\label{other-public-foundation}}

Public foundations are ``organizations that derive their funding or
support primarily from the general public in carrying our their social,
educational, religious or other charitable activities serving the common
welfare. Although public foundations may provide direct charitable
services to the public as other nonprofits do, their primary focus is on
grantmaking.''\footnote{({``IRS Activity Codes,''} n.d.)}

We can identify other non-501(c)(3) public foundations using SUBSECCD
(subsection code) and NTEECC (NTEECC primary purpose). Values of
anything other than 3 for SUBSECCD indicate that an organization is a
non-501(c)(3) organization, and values of T30 for NTEECC indicate that
an organization is a public foundation.

\hypertarget{c17-trust-providing-for-the-payment-of-supplemental-unemployment-compensation-benefits}{%
\subsection{501(c)(17) trust providing for the payment of supplemental
unemployment compensation
benefits}\label{c17-trust-providing-for-the-payment-of-supplemental-unemployment-compensation-benefits}}

An organization must meet the following requirements for to qualify for
exemption from federal income tax as a 501(c)(17) trust providing for
the payment of supplemental unemployment compensation benefits:

\begin{enumerate}
\def\labelenumi{\arabic{enumi}.}
\item
  ``The trust must constitute a valid trust under local law, be
  evidenced by an executed written document, and be part of a written
  plan.
\item
  The plan must be established and maintained by an employer or its
  employees solely for the purpose of providing supplemental
  unemployment compensation benefits.
\item
  The plan must provide that the corpus and income of the trust cannot
  be used for, or diverted to, any purpose other than providing such
  benefits prior to the satisfaction of all liabilities to the employees
  covered by the plan.
\item
  Benefits must be determined according to objective standards, and may
  not be determined solely in the discretion of the trustees.
\item
  The eligibility requirements and the benefits payable must not
  discriminate in favor of officers, shareholders, supervisory
  employees, or highly compensated employees.~ Benefits payable under
  the plan will not be considered discriminatory if they bear a uniform
  relationship to the total compensation of the employees covered by the
  plan.''\footnote{({``Supplemental Unemployment Benefits Trust -
    501(c)(17),''} n.d.)}
\end{enumerate}

These trusts apply for recognition of exemption from federal income tax
by filing Form 1024.\footnote{({``Supplemental Unemployment Benefits
  Trust - 501(c)(17),''} n.d.)} They must file an annual Form 990-N,
990-EZ, or 990 (Service 2023).

We can identify 501(c)(17) trusts providing for the payment of
supplemental unemployment compensation benefits using SUBSECCD
(subsection code). Values of 17 indicate that an organization is a
501(c)(17) trust providing for the payment of supplemental unemployment
compensation benefits.

\hypertarget{c18-employee-funded-pension-trust}{%
\subsection{501(c)(18) employee funded pension
trust}\label{c18-employee-funded-pension-trust}}

501(c)(18) employee funded pension trusts pay benefits under pension
plans that employees fund (Service 2023). These trusts apply for
recognition of exemption from federal income tax by filing Form
1024.\footnote{({``Supplemental Unemployment Benefits Trust -
  501(c)(17),''} n.d.)} They must file an annual Form 990-N, 990-EZ, or
990 (Service 2023).

We can identify 501(c)(18) employee funded pension trusts using SUBSECCD
(subsection code). Values of 18 indicate that an organization is a
501(c)(18) employee funded pension trust.

\hypertarget{c21-black-lung-benefit-trust}{%
\subsection{501(c)(21) black lung benefit
trust}\label{c21-black-lung-benefit-trust}}

Coal mine operators fund 501(c)(21) black lung benefit trusts ``to
satisfy their liability for disability or death due to black lung
diseases'' (Service 2023, 69). These trusts apply for recognition of
exemption from federal income tax by filing Form 1024.\footnote{({``Supplemental
  Unemployment Benefits Trust - 501(c)(17),''} n.d.)} They must file an
annual Form 990 (Service 2023).

We can identify 501(c)(21) black lung benefit trusts using SUBSECCD
(subsection code). Values of 21 indicate that an organization is a
501(c)(21) black lung benefit trust.

\hypertarget{a1-charitable-trust-treated-as-a-private-foundation}{%
\subsection{4947(a)(1) charitable trust treated as a private
foundation}\label{a1-charitable-trust-treated-as-a-private-foundation}}

A 4947(a)(1) charitable trusts is ``a trust that is not tax exempt, all
of the unexpired interests of which are devoted to one or more
charitable purposes, and for which a charitable contribution deduction
was allowed under a specific section of the Internal Revenue
Code.''\footnote{({``Charitable Trusts,''} n.d.)}

The IRS treats them as private foundations unless they meet one of the
requirements for being a 501(c)(3) public charity. This means the IRS
subjects them to the same taxes, restrictions, and requirements of
private foundations, which are more stringent than those it places on
public charities.\footnote{({``Charitable Trusts,''} n.d.)}

4947(a)(1) charitable trusts treated as private foundations must file an
annual Form 990-PF.\footnote{({``Instructions for Form 990-PF (2022),''}
  n.d.)}

We can identify 4947(a)(1) charitable trusts that are treated as private
foundations using SUBSECCD (subsection code). Values of 92 indicate that
an organization is a 4947(a)(1) charitable trust treated as a private
foundation.

\hypertarget{a1-charitable-trust-not-treated-as-a-private-foundation}{%
\subsection{4947(a)(1) charitable trust not treated as a private
foundation}\label{a1-charitable-trust-not-treated-as-a-private-foundation}}

A 4947(a)(1) charitable trusts is ``a trust that is not tax exempt, all
of the unexpired interests of which are devoted to one or more
charitable purposes, and for which a charitable contribution deduction
was allowed under a specific section of the Internal Revenue
Code.''\footnote{({``Charitable Trusts,''} n.d.)}

The IRS treats them as private foundations unless they meet one of the
requirements for being a 501(c)(3) public charity. This means that
4941(a)(1) charitable trusts that are not treated as private foundations
are not subjected to the same taxes, restrictions, and requirements of
private foundations, which are more stringent than those the IRS places
on public charities.\footnote{({``Charitable Trusts,''} n.d.)}

4947(a)(1) charitable trusts not treated as private foundations must
file an annual Form 990-EZ or 990.\footnote{({``Instructions for Form
  990-EZ (2022),''} n.d.; {``Instructions for Form 990 Return of
  Organization Exempt from Income Tax (2022),''} n.d.)}

We can identify 4947(a)(1) charitable trusts that are not treated as
private foundations using SUBSECCD (subsection code). Values of 91
indicate that an organization is a 4947(a)(1) charitable trust not
treated as a private foundation.

\hypertarget{a2-split-interest-charitable-trust}{%
\subsection{4947(a)(2) split-interest charitable
trust}\label{a2-split-interest-charitable-trust}}

4947(a)(2) split-interest charitable trusts ``make distributions to both
charitable and noncharitable beneficiaries, while providing tax benefits
to their donor.''\footnote{({``SOI Tax Stats - Split-Interest Trust
  Statistics,''} n.d.)} They are not exempt from federal income
tax.\footnote{({``Instructions for Form 5227 (2022),''} n.d.)}

4947(a)(2) split-interest charitable trusts must file an annual Form
5227: Split-Interest Trust Information Return.\footnote{({``Instructions
  for Form 5227 (2022),''} n.d.)}

We can identify 4947(a)(2) split-interest charitable trusts using
SUBSECCD (subsection code). Values of 90 indicate that an organization
is a 4947(a)(2) split-interest charitable trust.

\hypertarget{not-a-foundation-trust-or-grant-making-organization}{%
\subsection{Not a foundation, trust, or grant-making
organization}\label{not-a-foundation-trust-or-grant-making-organization}}

Organizations not described above are coded as ``not a foundation,
trust, or grant-making organization.''

\bookmarksetup{startatroot}

\hypertarget{references}{%
\chapter*{References}\label{references}}
\addcontentsline{toc}{chapter}{References}

\markboth{References}{References}

\hypertarget{refs}{}
\begin{CSLReferences}{1}{0}
\leavevmode\vadjust pre{\hypertarget{ref-irs-who-must-file}{}}%
{``Annual Exempt Organization Return: Who Must File.''} n.d. Internal
Revenue Service.
\url{https://www.irs.gov/charities-non-profits/annual-exempt-organization-return-who-must-file}.

\leavevmode\vadjust pre{\hypertarget{ref-irs-charitable-trusts}{}}%
{``Charitable Trusts.''} n.d. Internal Revenue Service.
\url{https://www.irs.gov/charities-non-profits/private-foundations/charitable-trusts}.

\leavevmode\vadjust pre{\hypertarget{ref-irs-def-of-exempt-op}{}}%
{``Definition of Exempt Operating Foundation.''} n.d. Internal Revenue
Service.
\url{https://www.irs.gov/charities-non-profits/private-foundations/definition-of-exempt-operating-foundation}.

\leavevmode\vadjust pre{\hypertarget{ref-irs-disqualified}{}}%
{``'Disqualified Individual' -- Exempt Operating Foundation.''} n.d.
Internal Revenue Service.
\url{https://www.irs.gov/charities-non-profits/private-foundations/disqualified-individual-exempt-operating-foundation}.

\leavevmode\vadjust pre{\hypertarget{ref-irs-exempt-op}{}}%
{``Exempt Operating Foundations.''} n.d. Internal Revenue Service.
\url{https://www.irs.gov/charities-non-profits/private-foundations/exempt-operating-foundations}.

\leavevmode\vadjust pre{\hypertarget{ref-irs-exempt-purposes}{}}%
{``Exempt Purposes -- Internal Revenue Code Section 501(c)(3).''} n.d.
Internal Revenue Service.
\url{https://www.irs.gov/charities-non-profits/charitable-organizations/exempt-purposes-internal-revenue-code-section-501c3}.

\leavevmode\vadjust pre{\hypertarget{ref-irs-grantmaking}{}}%
{``Grant-Making Foundations.''} n.d. Internal Revenue Service.
\url{https://www.irs.gov/charities-non-profits/private-foundations/grant-making-foundations}.

\leavevmode\vadjust pre{\hypertarget{ref-irs-form-5227}{}}%
{``Instructions for Form 5227 (2022).''} n.d. Internal Revenue Service.
\url{https://www.irs.gov/instructions/i5227}.

\leavevmode\vadjust pre{\hypertarget{ref-irs-instructions-8940}{}}%
{``Instructions for Form 8940 (04/2023).''} n.d. Internal Revenue
Service. \url{https://www.irs.gov/instructions/i8940}.

\leavevmode\vadjust pre{\hypertarget{ref-irs-990-instructions}{}}%
{``Instructions for Form 990 Return of Organization Exempt from Income
Tax (2022).''} n.d. Internal Revenue Service.
\url{https://www.irs.gov/instructions/i990}.

\leavevmode\vadjust pre{\hypertarget{ref-irs-990ez-instructions}{}}%
{``Instructions for Form 990-EZ (2022).''} n.d. Internal Revenue
Service. \url{https://www.irs.gov/instructions/i990ez}.

\leavevmode\vadjust pre{\hypertarget{ref-irs-990pf-instructions}{}}%
{``Instructions for Form 990-PF (2022).''} n.d. Internal Revenue
Service. \url{https://www.irs.gov/instructions/i990pf}.

\leavevmode\vadjust pre{\hypertarget{ref-activity-codes}{}}%
{``IRS Activity Codes.''} n.d. National Center for Charitable
Statistics. \url{https://nccs.urban.org/publication/irs-activity-codes}.

\leavevmode\vadjust pre{\hypertarget{ref-knuth84}{}}%
Knuth, Donald E. 1984. {``Literate Programming.''} \emph{Comput. J.} 27
(2): 97--111. \url{https://doi.org/10.1093/comjnl/27.2.97}.

\leavevmode\vadjust pre{\hypertarget{ref-irs-private-foundations}{}}%
{``Private Foundations.''} n.d. Internal Revenue Service.
\url{https://www.irs.gov/charities-non-profits/charitable-organizations/private-foundations}.

\leavevmode\vadjust pre{\hypertarget{ref-irs-private-op}{}}%
{``Private Operating Foundations.''} n.d. Internal Revenue Service.
\url{https://www.irs.gov/charities-non-profits/private-foundations/private-operating-foundations}.

\leavevmode\vadjust pre{\hypertarget{ref-irs-2014b}{}}%
Service, Internal Revenue. 2014. {``Form 1023-EZ: Streamlined
Application for Recognition of Exemption Under Section 501(c)(3) of the
Internal Revenue Code.''} Washington, DC.
\url{https://www.irs.gov/pub/irs-prior/f1023ez--2014.pdf}.

\leavevmode\vadjust pre{\hypertarget{ref-irs-2020a}{}}%
---------. 2020. {``Form 1023: Application for Recognition of Exemption
Under Section 501(c)(3) of the Internal Revenue Code.''} Washington, DC.
\url{https://www.pay.gov/public/form/preview/pdf/104}.

\leavevmode\vadjust pre{\hypertarget{ref-irs-2022f}{}}%
---------. 2022. {``Schedule a (Form 990): Public Charity Status and
Public Support.''} Washington, DC.
\url{https://www.irs.gov/pub/irs-pdf/f990sa.pdf}.

\leavevmode\vadjust pre{\hypertarget{ref-irs-2023c}{}}%
---------. 2023. {``Tax-Exempt Status for Your Organization.''}
Washington, DC. \url{https://www.irs.gov/pub/irs-pdf/p557.pdf}.

\leavevmode\vadjust pre{\hypertarget{ref-irs-split-interest-trust-stats}{}}%
{``SOI Tax Stats - Split-Interest Trust Statistics.''} n.d. Internal
Revenue Service.
\url{https://www.irs.gov/statistics/soi-tax-stats-split-interest-trust-statistics}.

\leavevmode\vadjust pre{\hypertarget{ref-irs-501c17}{}}%
{``Supplemental Unemployment Benefits Trust - 501(c)(17).''} n.d.
Internal Revenue Service.
\url{https://www.irs.gov/charities-non-profits/other-non-profits/supplemental-unemployment-benefits-trust-501c17}.

\end{CSLReferences}



\end{document}
